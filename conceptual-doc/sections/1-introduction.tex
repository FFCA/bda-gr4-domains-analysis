\section{Introduction:}\label{sec:introduction}
Domain information can change fast due to various reasons such as changes in ownership, or of the provider as well as of the servers themselves or their locations.
Different nameservers may be used concerning a domain space for, e.g. distribution of DNS-lookup load~\autocite[cf.][]{Nutter.2003} and hence involve multiple companies behind.
Therefore, information about these companies acting as providers is of interest within the topic so that they should be considered within the scope of the project.

Keeping track of these details in sense of current analyses in the field of related businesses areas requires the establishment of processes and strategies for information retrieval, processing, loading, and deployment.
Each of these steps should be realized using state-of-the-art techniques and technologies to enable continuous usage of the product.
Quality measures like code versioning and documentation are fundamental requirements that meet this demand.

This project aims at offering these characteristics and apply them to the field of big data analyses in the topic of domain information retrieved from DNS records having a valid SSL certificate at the time of the requests.
The results are available within a dashboard to visualize the addressed points of interest.